\documentclass[12pt]{beamer}
\usepackage{amsmath, amsfonts, amssymb, amsthm}
\usepackage{graphicx}
\usepackage[brazilian]{babel}
\usepackage[utf8]{inputenc}
\usepackage{verbatim}
\usepackage{hyperref}
\usetheme{Madrid}

\title{Mini-Curso de \LaTeX\ \\ Aula 05 - Fontes, Bibliografia, Links e Notas}
\author{Fábio Meneghetti \and Pedro Caetano}
\date{11 de abril de 2016}

\begin{document}
\begin{frame}
  \titlepage
\end{frame}

\begin{frame}{Licença}
  Esta apresentação está licenciada com uma Licença Creative Commons Atribuição-CompartilhaIgual 4.0 Internacional.
  \begin{center}
    \includegraphics[scale=0.3]{../license.png}
  \end{center}
\end{frame}

\begin{frame}
  \tableofcontents
\end{frame}

\begin{frame}{Fontes}
  \section{Fontes}
  \begin{itemize}
    \item Primeiramente, vamos ensinar a mudar a fonte padrão do \LaTeX! Como o padrão para fontes do \LaTeX\ (o METAFONT) é diferente do padrão da maioria dos editores de texto, nem todas as fontes estão disponíveis.
    \item Cada fonte tem um jeito específico de ser adicionada, e geralmente isso envolve adicionar pacotes extras.
    \item A página \url{www.tug.dk/FontCatalogue/} lista as fontes disponíveis no \LaTeX\ e explica como adicionar cada uma.
    \item Ver Exemplo 1!
  \end{itemize}
\end{frame}

\begin{frame}
  \begin{center}
    \large Obrigado!\\
    :)
  \end{center}
\end{frame}

\end{document}
