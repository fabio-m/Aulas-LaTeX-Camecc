\documentclass[12pt]{beamer}
\usepackage{amsmath, amsfonts, amssymb, amsthm}
\usepackage{graphicx}
\usepackage[brazilian]{babel}
\usepackage[utf8]{inputenc}
\usepackage{verbatim}
\usepackage{hyperref}
\usetheme{Madrid}

\title{Mini-Curso de \LaTeX\ \\ Aula 05 - Fontes, Bibliografia, Links e Notas}
\author{Fábio Meneghetti \and Pedro Caetano}
\date{11 de abril de 2016}

\begin{document}
\begin{frame}
  \titlepage
\end{frame}

\begin{frame}{Licença}
  Esta apresentação está licenciada com uma Licença Creative Commons Atribuição-CompartilhaIgual 4.0 Internacional.
  \begin{center}
    \includegraphics[scale=0.3]{../license.png}
  \end{center}
\end{frame}

\begin{frame}
  \tableofcontents
\end{frame}

\begin{frame}{Fontes}
  \section{Fontes}
  \begin{itemize}
    \item Primeiramente, vamos ensinar a mudar a fonte padrão do \LaTeX! Como o padrão para fontes do \LaTeX\ (o METAFONT) é diferente do padrão da maioria dos editores de texto, nem todas as fontes estão disponíveis.
    \item Cada fonte tem um jeito específico de ser adicionada, e geralmente isso envolve adicionar pacotes extras.
    \item A página \url{www.tug.dk/FontCatalogue/} lista as fontes disponíveis no \LaTeX\ e explica como adicionar cada uma.
    \item Ver Exemplo 1!
  \end{itemize}
\end{frame}

\begin{frame}{Bibliografia}
  \section{Bibliografia}
\end{frame}

\begin{frame}[fragile]{Footnotes}
  \section{Footnotes}

  Para adicionar notas de rodapé, basta usar o comando \verb+\footnote{}+ no lugar em que você quer que apareça o número, e com a nota como entrada.
\end{frame}

\begin{frame}[fragile]{Links}
  \section{Links}
  Para adicionar links (clicáveis!) no seu documento, basta adicionar o pacote \verb+hyperref+.\\
  
  A fim de introduzir no texto um endereço de internet que possua um link, usa-se o comando \verb+\url{http://pagina.web}+\\
  \medskip
  \textbf{Ex.:} 
  \begin{verbatim}  
  Na página \url{https://ctan.org/pkg/hyperref}
  podes encontrar a documentação do pacote
  hyperref.
  \end{verbatim}
\end{frame}

\begin{frame}[fragile]
  Já para introduzir em um trecho do texto um link a uma página específica, usa-se o comando \verb+\href{http://pagina.web}{texto}+\\
  \medskip
  \textbf{Ex.:}
  \begin{verbatim}
  \href{https://ctan.org/pkg/hyperref}{Clique aqui} 
  para acessar a documentação do pacote 
  hyperref.
  \end{verbatim}
  Outro efeito interessante do pacote \verb+hyperref+ é que referências cruzadas, itens do sumário e notas de rodapé passam a possuir links clicáveis no pdf produzido.
\end{frame}

\begin{frame}[fragile]{Teoremas}
  \section{Teoremas}
  Quando estamos digitando matemática em \LaTeX é comum necessitarmos de ambiente numerados. \\
  \medskip
  O exemplo mais comum são teoremas numerados, mas a situação é a mesma com definições, lemas, corolários etc... (veja ex2.pdf).\\
  \medskip
  Isto é feito com o comando \verb+\newtheorem{nome_do_ambiente}{Texto}+, do pacote \verb+amsthm+. 
\end{frame}

\begin{frame}[fragile]
  Note que o comando \verb+\newtheorem+ cria novos ambientes. Assim, se quisermos criar um ambiente \textit{teorema}, introduzimos no preâmbulo do documento o comando

  \begin{verbatim} 
\newtheorem{teorema}{Teorema }
  \end{verbatim}
  Sempre que quisermos criar um teorema no texto, invocamos este ambiente com a sintaxe usual
   
  \begin{verbatim} 
\begin{teorema} BlaBlaBla \end{teorema}
  \end{verbatim}
  e a saída aparecerá como 
  
  \bigskip  
  \textbf{Teorema 2} BlaBlaBla.

\end{frame}

\begin{frame}[fragile]
  Há também um ambiente para adicionar demonstrações à teoremas, o ambiente \textit{proof}, usado como
  
  \verb+\begin{proof} Prova a cargo do leitor \end{proof}+
  
  \bigskip
  
  \textbf{Dica: } Por padrão a numeração de ambientes tipo teorema é independente de seção, capítulo etc. Se quiser numerar teoremas no estilo (número do teorema).(número da seção) basta adicionar a opção \verb+section+ ao comando \verb+\newtheorem+

  \begin{verbatim} 
\newtheorem{teorema}{Teorema }[section] 
  \end{verbatim}
  
  Divirta-se com o exemplo ex2.tex!

\end{frame}

\begin{frame}
  \begin{center}
    \large Obrigado!\\
    :)
  \end{center}
\end{frame}

\end{document}
