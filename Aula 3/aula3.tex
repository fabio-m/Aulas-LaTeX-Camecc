\documentclass[12pt]{beamer}
\usepackage{amsmath, amsfonts, amssymb, amsthm}
\usepackage{graphicx}
\usepackage[brazilian]{babel}
\usepackage[utf8]{inputenc}
\usepackage{verbatim}
\usepackage{hyperref}
\usetheme{Madrid}

\title{Mini-Curso de \LaTeX\ \\ Aula 03 - Mais Formatação, Tabelas e Imagens}
\author{Fábio Meneghetti \and Pedro Caetano}
\date{4 de abril de 2016}

\begin{document}
\begin{frame}
  \titlepage
\end{frame}

\begin{frame}{Licença}
  Esta apresentação está licenciada com uma Licença Creative Commons Atribuição-CompartilhaIgual 4.0 Internacional.
  \begin{center}
    \includegraphics[scale=0.3]{../license.png}
  \end{center}
\end{frame}

\begin{frame}
  \tableofcontents
\end{frame}

\begin{frame}[fragile]{Alinhamento do texto}
  \section{Formatação (continuação)}
  O alinhamento pode ser feito usando um ambiente, ou um comando (é recomendado que se use o ambiente).
  \begin{center}
    \begin{tabular}{ccc}
    \textbf{Alinhamento} & \textbf{Ambiente} & \textbf{Comando}\\
    À esquerda & flushleft & \verb+\raggedright+\\
    À direita & flushright & \verb+\raggedleft+\\
    Centralizado & center & \verb+\centering+\\
    \end{tabular}
  \end{center}
\end{frame}

\begin{frame}[fragile]{Comandos para o texto}
  \begin{itemize}
    \item \textit{itálico}: \verb+\textit{texto}+
    \item \textbf{negrito}: \verb+\textbf{texto}+
    \item \underline{sublinhado}: \verb+\underline{texto}+
    \item \textsc{Letra de forma}: \verb+\textsc{texto}+
  \end{itemize}
\end{frame}

\begin{frame}[fragile]{Espaçamento}

\begin{itemize}
  \item \verb+\\+ - quebra para a próxima linha. Tem uma opção para um espaço extra com \verb+\\[2cm]+
  \item \verb+\pagebreak+ - quebra de página
  \item \verb+\bigskip+, \verb+\medskip+ e \verb+\smallskip+ pulam certos espaço
\end{itemize}

\end{frame}

\begin{frame}{Tabelas}
  \section{Tabelas}
\end{frame}

\begin{frame}
  \begin{center}
    \large Obrigado!\\
    :)
  \end{center}
\end{frame}

\end{document}
