\documentclass[12pt]{beamer}
\usepackage{amsmath, amsfonts, amssymb, amsthm}
\usepackage{graphicx}
\usepackage[brazilian]{babel}
\usepackage[utf8]{inputenc}
\usepackage{verbatim}
\usepackage{hyperref}
\usetheme{Madrid}

\title{Mini-Curso de \LaTeX\ \\ Aula 05 - Fontes, Bibliografia, Links e Notas}
\author{Fábio Meneghetti \and Pedro Caetano}
\date{25 de outubro de 2016}

\begin{document}
\begin{frame}
  \titlepage
\end{frame}

\begin{frame}{Licença}
  Esta apresentação está licenciada com uma Licença Creative Commons Atribuição-CompartilhaIgual 4.0 Internacional.
  \begin{center}
    \includegraphics[scale=0.3]{../license.png}
  \end{center}
\end{frame}

\begin{frame}
  \tableofcontents
\end{frame}

\begin{frame}{Fontes}
  \section{Fontes}
  \begin{itemize}
    \item Primeiramente, vamos ensinar a mudar a fonte padrão do \LaTeX! Como o padrão para fontes do \LaTeX\ (o METAFONT) é diferente do padrão da maioria dos editores de texto, nem todas as fontes estão disponíveis.
    \item Cada fonte tem um jeito específico de ser adicionada, e geralmente isso envolve adicionar pacotes extras.
    \item A página \url{www.tug.dk/FontCatalogue/} lista as fontes disponíveis no \LaTeX\ e explica como adicionar cada uma.
    \item Ver Exemplo 1!
  \end{itemize}
\end{frame}

\begin{frame}[fragile]{Bibliografia}
  \section{Bibliografia}
  Há várias formas de adicionar bibliografia no \LaTeX. A mais utilizada é com o programa BibTeX. Ele funciona da seguinte forma:
  \begin{itemize}
    \item Você cria um arquivo nome.bib, onde ficará uma lista de referências que você vai usar.
    \item Depois você coloca no fim do documento o código \verb+\bibliography{nome}+
    \item Aí basta citar essas refecências com o comando \verb+\cite{livro1}+, onde \verb+livro1+ é um nome definido para cada item no arquivo .bib
  \end{itemize}
\end{frame}

\begin{frame}[fragile]
  O arquivo .bib é simplesmente uma lista de itens da forma
  \begin{verbatim}
@book{ alice,
  title = {Alice No País Das Maravilhas},
  author = {Carroll, L.},
  isbn = {9780990339205},
  lccn = {2014908130},
  url = {books.google.com.br/books?id=DDjBoAEACAAJ},
  year = {2014},
  publisher = {Bilingualing LLC}
}
  \end{verbatim}
  onde o primeiro nome é o que você vai usar para referenciar o livro, e onde cada tipo de arquivo (book, article, etc.) tem opções diferentes para serem preenchidas.
\end{frame}

\begin{frame}[fragile]
  \begin{itemize}
    \item Uma tabela com todos os tipos de arquivo, e com todas as opções que cada tipo requer, pode se encontrada em \url{https://en.wikibooks.org/wiki/LaTeX/Bibliography_Management\#BibTeX}
    \item Geralmente as páginas de artigos científicos e de livros (como Google Books) já vem com um código de BibTeX pronto. (Vamos ver um exemplo na internet!)
    \item O seu editor de \LaTeX\ deve estar habilitado para rodar o BibTeX junto do \LaTeX\ (a maioria dos editores, eg. Texmaker, TexnicCenter, faz isso por padrão)
    \item Antes de colocar o \verb+\bibliography+, você pode colocar o comando \verb+\bibliographystyle{}+, com as entradas da tabela a seguir.
  \end{itemize}
\end{frame}

\begin{frame}
  \begin{table}[h]
    \small
    \begin{tabular}{|l|l|l|l|}
      \hline
      \textbf{Entrada} & \textbf{Formato de Nome} & \textbf{Referência} & \textbf{Ordem}\\
      \hline
      plain & Homer Simpson & Número & autor\\
      \hline
      unsrt & Homer Simpson & Número & referência\\
      \hline
      abbrv & H. Simpson & Número & autor\\
      \hline
      alpha & Homer Simpson & Sim95 & autor\\
      \hline
      abstract & Homer Simpson & Simpson-1995a & \\
      \hline
      acm & Simpson, H. & Número & \\
      \hline
      authordate1 & Simson, Homer & Simpson, 1995 & \\
      \hline
      apa & Simpson, H. (1995) & Simpson1995 & \\
      \hline
      named & Homer Simpson & Simpson 1995 & \\
      \hline
    \end{tabular}
  \end{table}
  (Tabela obtida de \url{https://en.wikibooks.org/wiki/LaTeX/Bibliography_Management\#BibTeX})
\end{frame}

\begin{frame}[fragile]{Footnotes}
  \section{Footnotes}

  Para adicionar notas de rodapé, basta usar o comando \verb+\footnote{}+ no lugar em que você quer que apareça o número, e com a nota como entrada.
\end{frame}

\begin{frame}[fragile]{Links}
  \section{Links}
  Para adicionar links (clicáveis!) no seu documento, basta adicionar o pacote \verb+hyperref+.\\

  A fim de introduzir no texto um endereço de internet que possua um link, usa-se o comando \verb+\url{http://pagina.web}+\\
  \medskip
  \textbf{Ex.:}
  \begin{verbatim}
  Na página \url{https://ctan.org/pkg/hyperref}
  podes encontrar a documentação do pacote
  hyperref.
  \end{verbatim}
\end{frame}

\begin{frame}[fragile]
  Já para introduzir em um trecho do texto um link a uma página específica, usa-se o comando \verb+\href{http://pagina.web}{texto}+\\
  \medskip
  \textbf{Ex.:}
  \begin{verbatim}
  \href{https://ctan.org/pkg/hyperref}{Clique aqui}
  para acessar a documentação do pacote
  hyperref.
  \end{verbatim}
  Outro efeito interessante do pacote \verb+hyperref+ é que referências cruzadas, itens do sumário e notas de rodapé passam a possuir links clicáveis no pdf produzido.
\end{frame}

\begin{frame}[fragile]{Teoremas}
  \section{Teoremas}
  Quando estamos digitando matemática em \LaTeX é comum necessitarmos de ambiente numerados. \\
  \medskip
  O exemplo mais comum são teoremas numerados, mas a situação é a mesma com definições, lemas, corolários etc... (veja ex2.pdf).\\
  \medskip
  Isto é feito com o comando \verb+\newtheorem{nome_do_ambiente}{Texto}+, do pacote \verb+amsthm+.
\end{frame}

\begin{frame}[fragile]
  Note que o comando \verb+\newtheorem+ cria novos ambientes. Assim, se quisermos criar um ambiente \textit{teorema}, introduzimos no preâmbulo do documento o comando

  \begin{verbatim}
\newtheorem{teorema}{Teorema }
  \end{verbatim}
  Sempre que quisermos criar um teorema no texto, invocamos este ambiente com a sintaxe usual

  \begin{verbatim}
\begin{teorema} BlaBlaBla \end{teorema}
  \end{verbatim}
  e a saída aparecerá como

  \bigskip
  \textbf{Teorema 2} BlaBlaBla.

\end{frame}

\begin{frame}[fragile]
  Há também um ambiente para adicionar demonstrações à teoremas, o ambiente \textit{proof}, usado como

  \verb+\begin{proof} Prova a cargo do leitor \end{proof}+

  \bigskip

  \textbf{Dica: } Por padrão a numeração de ambientes tipo teorema é independente de seção, capítulo etc. Se quiser numerar teoremas no estilo (número do teorema).(número da seção) basta adicionar a opção \verb+section+ ao comando \verb+\newtheorem+

  \begin{verbatim}
\newtheorem{teorema}{Teorema }[section]
  \end{verbatim}

  Divirta-se com o exemplo ex3.tex!

\end{frame}

\begin{frame}
  \begin{center}
    \large Obrigado!\\
    :)
  \end{center}
\end{frame}

\end{document}
