\documentclass[12pt, a4paper]{article}
\usepackage[utf8]{inputenc}
\usepackage{amsmath,amsthm,amssymb,amsfonts}
\usepackage[brazilian]{babel}

\title{Alice no País das Maravilhas}
\author{Lewis Carroll}
\date{4 de julho de 1865}

\begin{document}

  \maketitle

  Alice estava começando a ficar muito cansada de estar sentada ao lado de sua irmã e não ter nada para fazer: uma vez ou duas ela dava uma olhadinha no livro que a irmã lia, mas não havia figuras ou diálogos nele e “para que serve um livro”, pensou Alice, “sem figuras nem diálogos?”

  Então, ela pensava consigo mesma (tão bem quanto era possível naquele dia quente que a deixava sonolenta e estúpida) se o prazer de fazer um colar de margaridas era mais forte do que o esforço de ter de levantar e colher as margaridas, quando subitamente um Coelho Branco com olhos cor-de-rosa passou correndo perto dela.

  Não havia nada de muito especial nisso, também Alice não achou muito fora do normal ouvir o Coelho dizer para si mesmo “Oh puxa! Oh puxa! Eu devo estar muito atrasado!” (quando ela pensou nisso depois, ocorreu-lhe que deveria ter achado estranho, mas na hora tudo parecia muito natural); mas, quando o Coelho tirou um relógio do bolso do colete, e olhou para ele, apressando-se a seguir, Alice pôs-se em pé e lhe passou a idéia pela mente como um relâmpago, que ela nunca vira antes um coelho com um bolso no colete e menos ainda com um relógio para tirar dele. Ardendo de curiosidade, ela correu pelo campo atrás dele, a tempo de vê-lo saltar para dentro de uma grande toca de coelho embaixo da cerca.

\end{document}
