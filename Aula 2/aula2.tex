\documentclass[12pt]{beamer}
\usepackage{amsmath, amsfonts, amssymb, amsthm}
\usepackage{graphicx}
\usepackage[brazilian]{babel}
\usepackage[utf8]{inputenc}
\usepackage{verbatim}
\usepackage{hyperref}
\usetheme{Madrid}

\title{Mini-Curso de \LaTeX\ \\ Aula 02 - Formatação}
\author{Fábio Meneghetti \and Pedro Caetano}
\date{29 de março de 2016}

\begin{document}
\begin{frame}
  \titlepage
\end{frame}

\begin{frame}{Licença}
  Esta apresentação está licenciada com uma Licença Creative Commons Atribuição-CompartilhaIgual 4.0 Internacional.
  \begin{center}
    \includegraphics[scale=0.3]{../license.png}
  \end{center}
\end{frame}

\begin{frame}
  \tableofcontents
\end{frame}

\begin{frame}[fragile]{Página de Título}
  \section{Página de Título}

  É necessário adicionar algumas informações sobre o documento no cabeçalho.
  \begin{itemize}
    \item \verb+\title{Alice no País das Maravilhas}+
    \item \verb+\author{Lewis Carroll}+
    \item A data é adicionada automaticamente, mas para mudar, \verb+\date{4 de julho de 1865}+
  \end{itemize}

\end{frame}

\begin{frame}[fragile]
  O \LaTeX\ já monta automaticamente uma página de título usando as informações que nós demos.

  Para fazer isso, coloque no começo do documento o código \verb+\maketitle+ (ver Exemplo 1).
\end{frame}

\begin{frame}[fragile]
  Se, dentro das opções do comando \verb+\documentclass+ nós colocarmos a opção titlepage, ou seja,\\
  \medskip
  \verb+\documentclass[12pt,a4paper,titlepage]{article}+\\
  \medskip
  então o \LaTeX\ reserva uma página só para o título (tente fazer isso no Exemplo 1!).
\end{frame}

\begin{frame}[fragile]
  \textbf{Observações:}\\
  \begin{itemize}
    \item O comando para ir à próxima linha no \LaTeX\ é \verb+\\+, e ele deve ser usado para pular a linha dentro dos comando title, author e date.
    \item Para adicionar mais de um autor, eles devem ser separados por \verb+\and+
  \end{itemize}
\end{frame}

\begin{frame}[fragile]{Seções}
  \section{Seções}
  O \LaTeX\ já enumera as seções e subseções automaticamente.
  \begin{itemize}
    \item \verb+\section{}+ Adiciona uma seção com o nome introduzido
    \item \verb+\subsection{}+ Adiciona uma subseção
    \item \verb+\subsubsection{}+ Adiciona uma subsubseção
  \end{itemize}
  \bigskip
  Ver Exemplo 2.
\end{frame}

\begin{frame}[fragile]
  Para adicionar uma seção ou subseção não-numerada, adicione um asterisco no fim do comando.\\
  \medskip
  \textbf{Ex:} \verb+\section*{Introdução}+
  \medskip

  Tente no Exemplo 2!
\end{frame}

\begin{frame}[fragile]
  Para adicionar um sumário, utilize o comando \verb+\tableofcontents+
  \bigskip

  Tente no Exemplo 2 também!
\end{frame}

\begin{frame}{Formatação de texto}
  \section{Formatação de texto}
\end{frame}

\begin{frame}
  \begin{center}
    \large Obrigado!\\
    :)
  \end{center}
\end{frame}

\end{document}
