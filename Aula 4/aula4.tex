\documentclass[12pt]{beamer}
\usepackage{amsmath, amsfonts, amssymb, amsthm}
\usepackage{graphicx}
\usepackage[brazilian]{babel}
\usepackage[utf8]{inputenc}
\usepackage{verbatim}
\usepackage{hyperref}
\usetheme{Madrid}

\title{Mini-Curso de \LaTeX\ \\ Aula 04 - Matemática}
\author{Fábio Meneghetti \and Pedro Caetano}
\date{20 de outubro de 2016}

\begin{document}
\begin{frame}
  \titlepage
\end{frame}

\begin{frame}{Licença}
  Esta apresentação está licenciada com uma Licença Creative Commons Atribuição-CompartilhaIgual 4.0 Internacional.
  \begin{center}
    \includegraphics[scale=0.3]{../license.png}
  \end{center}
\end{frame}

\begin{frame}
  \tableofcontents
\end{frame}

\begin{frame}{Introdução}
  \section{Sobre o modo matemático}
  Para se adicionar equações e fórmulas matemáticas no \LaTeX, existe um ambiente dedicado a isso, com comandos específicos que só funcionam quando você está nesse ambiente. Além disso, todas as letras digitadas dentro desse ambiente serão colocadas em itálico, que é o padrão para variáveis matemáticas.
  \\[1cm]
  Mas existem várias formas de se entrar no ambiente matemático, e elas produzem resultados diferentes no documento. Vamos mostrá-las agora.
\end{frame}

\begin{frame}[fragile]{Entrando no modo matemático}
  \section{Entrando no modo matemático}
  \begin{enumerate}
    \item \textbf{O modo em linha (\textit{inline}):} Serve para adicionar matemática no meio de um texto normal. Para entrar nesse modo, basta colocar o código entre cifrões \$.

    \item \textbf{O modo display:} Serve para colocar uma equação matemática em linha própria, centralizada, dando certo destaque. Para entrar nesse modo, basta colocar o código entre \verb+\[+ e \verb+\]+.

    \item \textbf{Equações:} O resultado da equação é igual ao do modo display, porém ela se torna numerada. Desse forma, pode-se fazer referências a equações. Para criar uma equação, basta criar um ambiente equation (com \verb+\begin{equation}+ e \verb+\end{equation}+).

  \end{enumerate}
\end{frame}

\begin{frame}[fragile]
  Antes de fazer um exemplo, vamos aprender alguns comandos simples para poder brincar:
  \begin{itemize}
    \item Para se fazer o ``elevado'', basta utilizar o chapéu \verb+^+. \textbf{Ex:} \verb+x^2+ gera $x^2$
    \item Um valor subescrito é análogo, mas utilizando o \verb+_+. \textbf{Ex:} \verb-x_1 + x_2^3- gera $x_1 + x_2^3$
    \item Para adicionar mais de um só dígito no subescrito ou no elevado, basta ``encapsular'' tudo com chaves. \textbf{Ex:} \verb-e^{2 \pi i}- gera $e^{2 \pi i}$
  \end{itemize}
\end{frame}

\begin{frame}[fragile]
  \begin{itemize}
    \item A raiz é feita com o comando \verb+\sqrt{}+. A opção \verb+[n]+ pode ser adicionada para fazer a raiz $n$-ésima.
    \item Vamos fazer um exemplo! (Exemplo 1)
  \end{itemize}
\end{frame}

\begin{frame}{Símbolos}
  \section{Símbolos}

  A quantidade de símbolos disponíveis para se utilizar no modo matemático é extremamente grande, e não vamos gastar o tempo para mostrando um por um. Mas vamos passar alguns links de páginas com tabelas e listas de símbolos, para que você possa procurar aqueles que você deseja usar. Algumas páginas que julgamos serem boas são:

  \begin{itemize}
    \item \url{latex.wikia.com/wiki/List_of_LaTeX_symbols}
    \item \url{oeis.org/wiki/List_of_LaTeX_mathematical_symbols}
  \end{itemize}
\end{frame}

\begin{frame}[fragile]{Frações, integrais e somatórios}
  \section{Frações, integrais e somatórios}
  \begin{itemize}
    \item Para adicionar frações, o comando é \verb+\frac{numerador}{denominador}+
    \item O símbolo de integral é \verb+\int+. Ao adicionar valores no subescrito e no elevado da integral, esses valores são posicionados como os extremos de integração. \textbf{Ex:} \verb+\int_0^1 f(x) dx+ gera
    \[
    \int_{0}^1 f(x) dx
    \]
    \item \textbf{Exercício:} O símbolo de somatório é \verb+\sum+. Tente escrever no \LaTeX\ o seguinte somatório:
    \[
      \sum_{k=1}^n x^k
    \]
  \end{itemize}
\end{frame}

\begin{frame}{Matrizes}
  \section{Matrizes}
  \begin{itemize}
    \item As matrizes funcionam de forma muito similas às tabelas, porém no modo matemático.
    \item Há vários tipos de ambiente:
    \begin{itemize}
      \item pmatrix para matrizes com parênteses
      \item bmatrix para colchetes
      \item Bmatrix para chaves
      \item vmatrix para barras verticais $|$ (geralmente é o determinante)
      \item Vmatrix para barras verticais duplas $\|$
    \end{itemize}
  \end{itemize}
\end{frame}

\begin{frame}[fragile]
  \begin{itemize}
    \item Basta então adicionar as entradas exatamente como se fosse uma tabela, ou seja, separando as entradas com \verb+&+ e quebrando a linha com \verb+\\+
    \item \textbf{Dica:} Para matrizes muito grandes, pode ser bom usar reticências:
    \begin{itemize}
      \item Horizontais: \verb+\cdots+ gera $\cdots$
      \item Verticais: \verb+\vdots+ gera $\vdots$
      \item Diagonais: \verb+\ddots+ gera $\ddots$
    \end{itemize}
    \item Vamos ver o Exemplo 2!
  \end{itemize}

\end{frame}

\begin{frame}[fragile]{O ambiente align}
  \section{Align}
  \begin{itemize}
    \item Às vezes você precisa escrever várias equações, uma em cima da outra, mas elas possuem cada uma um tamanho diferente, de tal forma que elas não vão ficar alinhadas. Uma solução é o ambiente matemático align, que alinha suas equações.
    \item Para alinhas as equações dentro de um ambiente align, basta separar o lado esquerdo e o direito com um \verb+&+, e usar um \verb+\\+ para terminar a linha.
    \item Dá também para usar vários \verb+&+ para fazer vários alinhamentos. Vamos ver o Exemplo 3!
  \end{itemize}

\end{frame}

\begin{frame}[fragile]{Algumas coisainhas a mais}
  \begin{itemize}
    \item Para adicionar uma lista de equações precididas por uma chave, para indicar ``casos'', basta usar o ambiente cases e dividir as equações usando \verb+\\+
    \item Para adicionar um texto no meio do modo matemático, basta usar o comando \verb+\text{}+, ou se for só uma palara, \verb+\mathrm{}+
    \item Para colocar parênteses ou colchetes em uma expressão grande, e fazer que o tamanho dos parênteses se ajustem ao tamanho da expressão, basta usar os comandos \verb+\left(+ e \verb+\right)+ (ou \verb+[+ e \verb+]+)
  \end{itemize}
\end{frame}

\begin{frame}[fragile]
  \begin{itemize}
    \item Alguns comandos como \verb+\sin+, \verb+\cos+, \verb+\lim+, \verb+\log+ estão definidos no \LaTeX, para não serem exibidos em itálico, e no caso de \verb+\lim+, para se adicionar o subescrito \verb+x \to \infty+ embaixo
    \item Há muitos recursos no modo matemático, e tentamos falar sobre o máximo possível deles, mas não dá pra falar de tudo. Dê uma olhada em \url{en.wikibooks.org/wiki/LaTeX/Mathematics},
    pois a página está bem completa.
    \item Veja o Exemplo 4!
  \end{itemize}
\end{frame}

\begin{frame}
  \begin{center}
    \large Obrigado!\\
    :)
  \end{center}
\end{frame}

\end{document}
