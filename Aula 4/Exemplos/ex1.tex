\documentclass[12pt, a4paper]{article}
\usepackage[utf8]{inputenc}
\usepackage[brazilian]{babel}
\usepackage{amsmath, amssymb, amsthm, amsfonts}

\title{Algumas equações matemáticas}
\author{John B. Fraleigh}
\date{}

\begin{document}

  \maketitle

  Considere a função polinomial
  \[
    p(x) = x^2 + x - 2
  \]
  que tem como zeros os valores $x_1 = 1$ e $x_2 = -2$. Então obtemos que a função também pode ser escrita na forma
  \begin{equation}
    p(x) = (x-1)(x+2).
  \end{equation}
\end{document}
