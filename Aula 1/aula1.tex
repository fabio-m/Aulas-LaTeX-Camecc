\documentclass[12pt]{beamer}
\usepackage{amsmath, amsfonts, amssymb, amsthm}
\usepackage{graphicx}
\usepackage[brazilian]{babel}
\usepackage[utf8]{inputenc}
\usepackage{verbatim}
\usepackage{listings}
\usepackage{hyperref}
\usetheme{Warsaw}

\title{Mini-Curso de \LaTeX\ \\ Aula 01}
\author{Fábio Meneghetti,\\
Pedro Caetano}
\date{}

\begin{document}
  \begin{frame}
    \titlepage
  \end{frame}

\begin{frame}{Licença}
    Esta apresentação está licenciada com uma Licença Creative Commons Atribuição-CompartilhaIgual 4.0 Internacional.
    \begin{center}
      \includegraphics[scale=0.3]{../license.png}
    \end{center}
\end{frame}

  \begin{frame}{Introdução}
    \section{Introdução}
    \begin{large}O que é \LaTeX?\end{large}

    \begin{itemize}
      \item Um processador de documentos (como o Word)
      \item Possui mais recursos para fazer fórmulas matemáticas
      \item Você tem mais controle da diagramação, do formato e do visual
      \item É totalmente livre
    \end{itemize}

  \end{frame}

  \begin{frame}
    \frametitle{Exemplos}
    Fábio, mostre exemplos de documentos feitos no \LaTeX, para motivar os alunos!
  \end{frame}

\begin{frame}{Instalação}
    \begin{itemize}
      \item Windows: \url{http://miktex.org/}
      \item Mac OS X: \url{https://www.tug.org/mactex/}
      \item GNU/Linux: \url{https://www.tug.org/texlive/}
    \end{itemize}
    Também é bom baixar um ambiente de desenvolvimento integrado. Recomendamos o Texmaker (\url{http://www.xm1math.net/texmaker/}).
\end{frame}

\begin{frame}[fragile]
  \section{Comandos}
    \frametitle{Comandos}
     Todos os comandos começam com a ``barra invertida'' \verb+\+
\medskip

     Tipos de comandos:
     \begin{itemize}
       \item Comandos diretos. \textbf{Ex:} \verb+\S+ $\rightarrow$ \S

       \item Comandos com entrada. \textbf{Ex:} \verb+\underline{Olá!}+ $\rightarrow$ \underline{Olá!}

     \end{itemize}
\end{frame}

\begin{frame}[fragile]
  \begin{itemize}
    \item Comandos com entrada e opões. \textbf{Ex:} \verb+\usepackage[brazilian]{babel}+

    \item Ambientes. \textbf{Ex:}
    \begin{verbatim}
    \begin{enumerate}
      \item item 1
      \item segundo item
    \end{enumerate}
  \end{verbatim}
  \begin{itemize}
    \item item 1
    \item segundo item
  \end{itemize}
  \end{itemize}
\end{frame}

\begin{frame}{Hello World!}
  \section{Hello World!}
  \large Vamos fazer nosso primeiro documento em \LaTeX!
  Mas primeiro, precisamos aprender alguns comandos básicos.
\end{frame}

\begin{frame}[fragile]
  \begin{center}\large
    \verb+\documentclass[...]{...}+
  \end{center}
  Dá informações sobre o tipo de documento que vamos usar.
  \begin{itemize}
    \item A entrada pode ser \underline{article}, beamer, book, report, letter, etc.
    \item As opções podem ser \underline{12pt}, \underline{a4paper}, notitlepage, twocolumn, landscape, etc.
  \end{itemize}
\end{frame}

\end{document}
