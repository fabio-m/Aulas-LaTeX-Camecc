\documentclass[12pt]{beamer}
\usepackage{amsmath, amsfonts, amssymb, amsthm}
\usepackage{graphicx}
\usepackage[brazilian]{babel}
\usepackage[utf8]{inputenc}
\usepackage{verbatim}
\usepackage{hyperref}
\usetheme{Madrid}

\title{Mini-Curso de \LaTeX\ \\ Aula 01 - Introdução}
\author{Fábio Meneghetti \and Pedro Caetano}
\date{11 de outubro de 2016}

\begin{document}
\begin{frame}
  \titlepage
\end{frame}

\begin{frame}{Licença}
  Esta apresentação está licenciada com uma Licença Creative Commons Atribuição-CompartilhaIgual 4.0 Internacional.
  \begin{center}
    \includegraphics[scale=0.3]{../license.png}
  \end{center}
\end{frame}

\begin{frame}
  \tableofcontents
\end{frame}

\begin{frame}{Introdução}
  \section{Introdução}
  \begin{large}O que é \LaTeX?\end{large}

  \begin{itemize}
    \item Um processador de documentos (como o Word)
    \item Possui mais recursos para fazer fórmulas matemáticas
    \item Você tem mais controle da diagramação, do formato e do visual
    \item É totalmente livre
  \end{itemize}

\end{frame}

\begin{frame}
  \frametitle{Exemplos}
  Mostre exemplos de documentos feitos no \LaTeX, para motivar os alunos!
\end{frame}

\begin{frame}{Instalação}
  \section{Instalação}
  \begin{itemize}
    \item Windows: \url{http://miktex.org/}
    \item Mac OS X: \url{https://www.tug.org/mactex/}
    \item GNU/Linux: \url{https://www.tug.org/texlive/}
  \end{itemize}
  Também é bom baixar um ambiente de desenvolvimento integrado. Recomendamos o Texmaker (\url{http://www.xm1math.net/texmaker/}).
\end{frame}

\begin{frame}[fragile]
  No \LaTeX, você primeiro escreve o código do documento em um arquivo de texto (com a extensão .tex), e depois você \emph{compila} esse código, ou seja, transforma ele em um PDF.

  \begin{itemize}
    \item Pela linha de comando:\\
    \verb+$ pdflatex arquivo.tex+
    \item Pelo ambiente de desenvolvimento (Texmaker): basta clicar no botão de compilar e ele vai criar um PDF na mesma pasta do arquivo
  \end{itemize}
\end{frame}

\begin{frame}[fragile]
  \section{Comandos}
  \frametitle{Comandos}
  Todos os comandos começam com a ``barra invertida'' \verb+\+
  \medskip

  Tipos de comandos:
  \begin{itemize}
    \item Comandos diretos. \textbf{Ex:} \verb+\S+ $\rightarrow$ \S

    \item Comandos com entrada. A entrada deve ser adicionada entre chaves \verb+{}+ imediatamente após o comando \textbf{Ex:} \verb+\underline{Olá!}+ $\rightarrow$ \underline{Olá!}

    \item Ambientes, que começam com \verb+\begin{nome_do_ambiente}+ e terminam com \verb+\end{nome_do_ambiente}+ \textbf{Ex:} Ambientes de tabelas, imagens, equações, etc.
  \end{itemize}

\end{frame}

\begin{frame}[fragile]
  Alguns comandos podem receber opções extras, e isso é feito adicionando colchetes \verb+[]+ antes das chaves.
  \medskip

  \textbf{Ex:} Para adicionar pacotes (ver slide \ref{pacotao}):
    \begin{verbatim}
      \usepackage[brazilian]{babel}
    \end{verbatim}
\end{frame}

\begin{frame}{Hello World!}
  \section{Hello World!}
  \large Vamos fazer nosso primeiro documento em \LaTeX!
  Mas primeiro, precisamos aprender alguns comandos básicos.
\end{frame}

\begin{frame}
  Primeiramente, é bom saber que o código é dividido em duas partes: o \underline{cabeçalho} e o \underline{corpo} do documento.

  \begin{itemize}
  \item O cabeçalho dá informações sobre o documento e sobre sua forma, e não sobre o conteúdo em si.

  \item O corpo é onde colocamos todo o conteúdo que queremos que seja exibido no documento.
  \end{itemize}

  Apresentaremos eles a seguir.
\end{frame}

\begin{frame}[fragile]
  \begin{center}\large
    \verb+\documentclass[...]{...}+
  \end{center}
  Dá informações sobre o tipo de documento que vamos usar.
  \begin{itemize}
    \item A entrada pode ser \underline{article}, beamer, book, report, letter, etc.
    \item As opções podem ser \underline{12pt}, \underline{a4paper}, titlepage, twocolumn, landscape, etc.
  \end{itemize}
\end{frame}

\begin{frame}[fragile]\label{pacotao}
  \begin{center}\large
    \verb+\usepackage{...}+
  \end{center}
  Para inserir novos recursos no \LaTeX.
  \begin{itemize}
    \item Matemática: \verb+amsmath, amsfonts, amssymb, amsthm+
    \item Acentos e til: \verb+\usepackage[utf8]{inputenc}+
    \item pt-BR: \verb+\usepackage[brazilian]{babel}+
  \end{itemize}
\end{frame}

\begin{frame}[fragile]
  \begin{center}\large
    Ambiente \verb+document+ (este é o corpo)
  \end{center}
  É o ambiente dentro do qual todo o documento que você escreve se encontra!
  \begin{verbatim}
    \begin{document}
      ...
    \end{document}
  \end{verbatim}
\end{frame}

\begin{frame}
  \begin{center}\large
    Vamos agora fazer o Hello World! do \LaTeX!
  \end{center}
\end{frame}

\begin{frame}[fragile]
\begin{verbatim}
\documentclass[12pt, a4paper]{article}
\usepackage[utf8]{inputenc}
\usepackage{amsmath,amsthm,amssymb,amsfonts}
\usepackage[brazilian]{babel}

\begin{document}

  Olá \LaTeX!

\end{document}
\end{verbatim}
\end{frame}

\begin{frame}[fragile]{Buscando ajuda}
  Às vezes acontecem erros estranhos durante a compilação, ou você não consegue se lembrar quais são as opções de algum pacote que você precisa, por exemplo. Neste caso, algumas fontes úteis de ajuda são
  \begin{itemize}
	\item a apostila do Raniere (na página do curso)
  \item um mecanismo de pesquisa (DuckDuckGo, Google, etc)
  \item o TeX Stack Exchange (\url{https://tex.stackexchange.com/})
  \item o site do Comprehensive \TeX\  Archive Network (\url{https://ctan.org/}), onde você pode provavelmente pode encontrar a documentação do pacote que está utilizando e\\
  \item o Wikibook do \LaTeX\ (\url{https://en.wikibooks.org/wiki/LaTeX}).
  \end{itemize}

\end{frame}

\begin{frame}
  \begin{center}
    \large Obrigado!\\
    :)
  \end{center}
\end{frame}

\end{document}
